\chapter{Internet vecí}\label{Internet_veci}
Bakalárska práca je založená na myšlienke a technológiách \emph{internetu vecí}, preto je mu venovaná táto kapitola. 
V~podkapitole \ref{iot-history} je obsiahnutá krátka história \emph{internetu vecí} (ďalej už len skratka IoT z~anglického \emph{internet of things}), definícia čo IoT v~dnešnej dobe predstavuje je v~podkapitole \ref{iot-definition} a podkapitola \ref{iot-smartcity} sa venuje IoT v~rozsahu celých miest. 
V~podkapitole \ref{Iot-technology} sú zhrnuté základné technológie používané v~IoT.

\section{História}\label{iot-history}
Napriek pár zariadeniam, ktoré boli už skôr pripojené na internet, bolo pravdepodobne IoT prvýkrát spomenuté iba ako názov prezentácie z~roku 1999, ktorej autorom bol Kevin Ashton~\cite{ashton2009internet}. 
V~tom čase boli takmer všetky dáta na internete vytvorené a nahraté ľuďmi. 
Jeho myšlienkou bolo zapojiť do internetu nielen ľudí ale aj predmety. 
Tie mali byť schopné komunikovať a zdieľať  o~sebe informácie bez ľudskej interakcie. 
Tým chcel dosiahnuť, aby sme mohli všetko sledovať, spočítať a tak výrazne znížiť odpad, straty a náklady \cite{ashton2009internet,patel2016internet,IoT-History}. 
Pre tento koncept ale bolo vytvorených mnoho definícií a názvov; niektorý ho nazývali \emph{Internet~vecí}, iný využili názvy komunikácia stroj-stroj z~anglického \emph{Machine to machine} (M2M). 
Potom sa použili aj názvy ako inteligentné služby a dokonca priemyselný internet \cite{10.5555/2785661}.
Každý túto myšlienku pomenoval~a~zadefinoval len čiastočne~a~to z~pohľadu, ktorý je mu najbližší~\cite{kellmereit2013silent}.



\section{Definícia}\label{iot-definition}
Ako bolo popísané v~minulej podkapitole \ref{iot-history}, názvov a definícii je veľa. Všetky majú niečo spoločné a dá sa z~nich vyvodiť jedná univerzálnejšia definícia.

Podľa zdroja \cite{keramidas2016components}, \textit{IoT popisuje fyzické objekty, alebo skupinu takýchto objektov, so senzormi, softvérom, schopnosťou spracovania dát a pripojením na internet. 
Tieto predmety zbierajú, spracovávajú, odosielajú dáta a komunikujú medzi sebou, inými systémami a v~nejakej komunikačnej sieti.} 
Fyzickými objektmi sú myslené teplomery, termostaty, bezpečnostné zámky, kamery, pohybové senzory, automaty, osvetlenia a veľa ďalších. 

Pre porovnanie podľa zdroja \cite{Shafiq2022} sa dá IoT definovať ako \emph{sieť fyzických zariadení, vozidiel, domácich spotrebičov a iných zariadení so vstavanou elektronikou, softwarom, senzormi, akčnými členmi a pripojením, ktoré týmto objektom umožňuje spojiť sa a vymieňať si dáta.} 
Ďalšia podobná definícia podľa \cite{cambridge} definuje IoT ako \emph{Objekty s~počítačovými zariadeniami, ktoré sa môžu navzájom pripojiť a vymieňať si údaje prostredníctvom internetu.}




\section{Inteligentné mesto}\label{iot-smartcity}
Jedným z~mojich cieľov bakalárskej práce je doniesť jednu službu inteligentných miest aj~priamo do domácností ľudí. 
Jedná sa o~inteligentné vykurovanie, ktoré tieto inteligentné mestá využívajú.

\subsection{Urbanizácia}
Urbanizácia svetového obyvateľstva sa stala veľkým problémom, ktorý treba riešiť. 
V~50.~rokoch minulého storočia žilo v~mestách len 30\% svetovej populácie, do roku 2014 dosiahla úroveň urbanizácie 54\% a Organizácia Spojených Národov predpokladá, že do roku 2050 bude toto číslo 66\% \cite{gerland2014world}. 
Proces urbanizácie výrazne zlepšil životnú úroveň ľudí, pričom zabezpečil dodávky vody a kanalizačné systémy, obytné a kancelárske budovy, vzdelávacie a zdravotnícke služby a pohodlnú dopravu \cite{davis1965urbanization}. 
Mestá sú zvyčajne regionálnymi hospodárskymi centrami, ktoré pomáhajú zlepšovať regionálnu hospodársku prosperitu a vytvárať viac pracovných miest. 
Koncentrácia vzdelaných ľudí v~mestách pomáha zlepšovať priemyselnú štruktúru a podporovať efektívnosť výroby \cite{bertinelli2004urbanization}.

Urbanizácia však vytvára aj nové výzvy a problémy. 
Rastúca populácia a maximálne využívanie prírodných zdrojov v~mestách spôsobuje ekologické a environmentálne problémy a zvyšuje problémy verejného neporiadku. 
Presnejšie rozprávame o~preľudnení, znečistení vzduchu a vody, degradácii životného prostredia, šírení nákazlivých chorôb a trestnej činnosti~\cite{haughton1997developing}. 
Všetky tieto výzvy a problémy nútia občanov, vlády a zainteresované strany venovať pozornosť životnému prostrediu a trvalo udržateľnému rozvoju miest a pokúsiť sa nájsť technické riešenia na zníženie týchto mestských problémov.
Revolúcia informačných a~komunikačných technológií (ďalej už len používaná skratka IKT) ponúkla ľuďom príležitosť znížiť rozsah, ba až vyriešiť otázky urbanizácie \cite{Yin2015}.

Urbanizácia, rast a súvisiace problémy moderných miest spolu s~rýchlym rozvojom nových IKT nám umožnili najprv si predstaviť koncepciu \emph{inteligentných miest} a teraz aj začať budovať inteligentné mestá, ktoré sa považujú za budúcu formu miest \cite{Yin2015}.

\subsection{Charakteristika}
Inteligentné mesto je technologicky moderná mestská oblasť, ktorá používa rôzne typy elektronických metód a snímačov na zhromažďovanie konkrétnych údajov. 
Informácie získané z~týchto údajov sa používajú na efektívne riadenie aktív, zdrojov a služieb; na oplátku sa tieto údaje používajú na zlepšenie prevádzky v~meste \cite{goldsmith2021define}. 
To zahŕňa údaje zozbierané od občanov, zariadení, budov a aktív, ktoré sa spracúvajú a analyzujú na účely monitorovania a riadenia dopravných systémov, elektrární, verejných služieb, vodohospodárskych sietí, odpadu, trestných vyšetrovaní, informačných systémov, škôl, knižníc, nemocníc a iných spoločenských služieb. 
V~inteligentných mestách je zdieľanie údajov nielen obmedzené na~samotné mesto, ale zahŕňa aj podniky, občanov a iné tretie strany, ktoré môžu tieto údaje rôzne využívať. 
Výmena údajov z~rôznych systémov a sektorov vytvára príležitosti na väčšie porozumenie a hospodárske výhody \cite{paiho2022opportunities}.

Koncept inteligentného mesta integruje IKT a rôzne fyzické zariadenia pripojené k~IoT s~cieľom optimalizovať efektívnosť mestských operácií, služieb a dostať sa do bližšieho kontaktu s~občanom \cite{gracia2018sustainable}. 
Technológia inteligentného mesta umožňuje mestským úradníkom priamo komunikovať s~komunitnou a mestskou infraštruktúrou a monitorovať, čo sa v~meste deje a~ako sa mesto vyvíja. 
Na riadenie toku mesta a umožnenie reakcií v~reálnom čase sa vyvíjajú inteligentné mestské aplikácie \cite{komninos2013makes}. 
Inteligentné mesto môže byť preto pripravenejšie reagovať na rôzne výzvy ako mesto s~konvenčným \emph{transakčným} vzťahom so svojimi občanmi.

\section{Technológie}\label{Iot-technology}
V~tejto časti sa venujem pár vybraným technológiám, ktoré som skúmal pre účely vyhotovenia práce. 
Technológie IoT je možné rozdeliť do troch časti podľa toho, čo robia s~dátami. 
Jedná sa o~zbieranie dát opísané v~podkapitole \ref{iot-colection}, prenos dát opísaný v~podkapitole \ref{iot-trasport} a uloženie a spracovanie dát opísané v~podkapitole \ref{iot-data}.


\subsection{Zber dát}\label{iot-colection}
Zhromažďovanie údajov IoT je proces používania senzorov na sledovanie stavu fyzických vecí. 
Zariadenia a technológie pripojené cez IoT dokážu monitorovať a merať dáta v~reálnom čase. 
Takéto senzory sú napríklad: senzory priblíženia, pozície, okupácie, pohybu, rýchlosti, teploty, tlaku, vlhkosti, kvality vody, chemické, magnetické a mnoho iných.~\cite{8862778}

Mnou potrebné senzory sú najmä senzory teploty a vlhkosti. 
Nakoľko tieto faktory zohrávajú najväčšiu rolu pri vyhodnocovaní vnútornej pocitovej teploty~\cite{AUniversalScaleofApparentTemperature}. 

Ďalej je pri výbere senzorov podstatný spôsob, akým senzor ďalej propaguje namerané hodnoty. 
K~tomu sa v~IoT používa mnoho prenosových spôsobov. 
Môžu byť napríklad jednoducho pripojené káblom alebo komunikovať na bezdrôtovej, rádiovej báze. 
Viac o~niekoľkých protokoloch a technológiách zabezpečujúcich bezdrôtovú komunikáciu IoT v~podkapitole~\ref{iot-trasport}.


\subsection{Prenos dát}\label{iot-trasport}
V~tejto podkapitole sú priblížené štyri protokoly, ktoré boli zvažované pre riešenie tejto práce:
\begin{itemize}
    \item \textit{\textbf{Wi-fi}}\footnote{Skratka pre anglické \emph{wireless fidelity}, čo znamená \emph{bezdrôtová vernosť}.} je najpoužívanejší a najznámejší protokol bezdrôtovej komunikácie, ktorý bol založený na štandarde IEEE 802.11. 
    To je podporované už existujúcou káblovou architektúrov, typicky \emph{Ethernet}. 
    Jeho široké využitie v~sektore IoT je sťažované vyššou než priemernou spotrebou energie, čo je spôsobené vyšším vyžarovacím výkonom. 
    To~s~frekvenciou 2.4 alebo 5 GHz. 
    Ten je nutný pre rýchly prenos dát vyšších objemov a lepšie pripojenie a spoľahlivosť s~nízkou latenciou.~\cite{elkhodr2016emerging}
    
    \item \textit{\textbf{Bluetooth LE}} (ďalej používaná skratka BLE) je vylepšenie klasického \emph{Bluetooth} protokolu. 
    Pracuje na frekvencii 2,4 Ghz. 
    Má mnoho aplikácií, ale jednou z~najbežnejších je prenos údajov zo senzorov. 
    Senzorové zariadenie, ktoré meria teplotu raz za minútu, alebo GPS zariadenie, ktoré zaznamenáva a vysiela svoju polohu každých 10 minút, je niekoľko príkladov. 
    V~mnohých prípadoch sú produkty BLE napájané iba z~malej gombíkovej batérie. 
    Ak sa údaje odosielajú len zriedka, zariadenie BLE napájané z~gombíkovej batérie môže mať životnosť batérie rok alebo dlhšie.~\cite{elkhodr2016emerging}

    To vytvára príležitosti pre mnoho IoT aplikácii hlavne medzi mobilnými a \emph{wearable} zariadeniami.
    BLE technológia je obmedzovaná hlavne limitovaným dosahom a pokrytím.~\cite{s17071467}\newpage

    \item \textit{\textbf{Zigbee}} je ďalšia sieťová technológia krátkeho dosahu podobná v~mnohých ohľadoch Bluetooth LE s~podobnými aplikáciami. 
    Používa rovnakú nosnú frekvenciu 2,4 GHz, spotrebúva veľmi málo energie, pracuje v~podobnom rozsahu a ponúka \textit{mesh networking}~\cite{elkhodr2016emerging}. 
    V~skutočnosti, Zigbee \textit{mesh network} môže obsahovať až 65 000 zariadení, čo je vyše dvakrát toľko, ako podporuje Bluetooth LE. 
    Zigbee sa primárne používa pre aplikácie domácej automatizácie, ako sú inteligentné osvetlenie, inteligentné termostaty a monitorovanie domácej energie. 
    Bežne sa používa aj v~priemyselnej automatizácii, inteligentných meračoch a bezpečnostných systémoch.~\cite{5942102}

    \item \textit{\textbf{LoRa}}\footnote{Význam skratky je z~anglického \emph{Long-Range}, čo znamená dlhý dosah.} umožňuje v~niektorých oblastiach komunikáciu s~veľmi dlhým dosahom viac ako 10~km pri nízkej spotrebe energie. 
    Ide o~patentovanú bezdrôtovú technológiu získanú spoločnosťou \emph{Semtech} v~roku 2012~\cite{slats_2021}.
    LoRa využíva rôzne frekvenčné pásma v~závislosti od regiónu prevádzky. 
    V~Severnej Amerike sa používa 915 MHz a v~Európe je frekvencia 868 MHz. 
    Iné oblasti môžu tiež používať 169 MHz a 433 MHz. 
    LoRa odkazuje na základnú technológiu a možno ju priamo použiť na komunikáciu typu \textit{peer-to-peer}. 
    LoRaWAN označuje sieťový protokol vyššej vrstvy.~\cite{elkhodr2016emerging}

    Hoci je LoRa navrhnutá tak, aby fungovala v~obrovskom rozsahu, nie je to mobilná technológia, ktorá sa môže pripojiť k~mobilným sieťam. 
    Vďaka tomu je menej zložitá a lacnejšia na implementáciu, ale jej aplikácie sú obmedzené. 
    Ak napríklad produkt vyžaduje vzdialený prístup ku cloudu, je potrebné dodať aj zariadenie LoRa brány na~pripojenie k~internetu. 
    Zariadenie sa pripája k~internetu a komunikuje s~akýmikoľvek vzdialenými LoRa zariadeniami. 
    LoRa neposkytuje žiadnemu jednému vzdialenému zariadeniu metódu vzdialeného prístupu ku cloudu, za predpokladu, že v~prevádzkovom dosahu nie je žiadna LoRa brána. 
    LoRa je ďalej limitovaná nízkou rýchlosťou prenosu\footnote{Táto rýchlosť dosahuje iba po \emph{27 kbps}.}, pracovnými cyklami v~LoRa sieti, čo limituje počet správ poslaných v~určitom časovom okne.
    Preto je LoRa nevhodná pre aplikácie potrebujúce nízku latenciu.~\cite{8474715}

    \item \textit{\textbf{IQRF}} je podľa zdroja \cite{IQRF} technológia založená na protokole IQRF, ktorý pracuje v~rádiovom pásme 868~MHz (v~Európe) alebo 915~MHz (v~USA a iných častiach sveta). To poskytuje vysokú priepustnosť dát a dobrý dosah, čo znamená, že zariadenia môžu byť umiestnené aj na relatívne veľkých vzdialenostiach.

    Zdroj \cite{IQRF} ďalej ukazuje, že IQRF podporuje rôzne topológie sieti vrátane hviezdicovej, meshovanej a rozptýlenej. Táto technológia tiež podporuje viaceré druhy zariadení, vrátane senzorov, aktuátorov a brán, ktoré môžu byť pripojené k~sieti a komunikovať s~inými zariadeniami a uzlami.
    Jednou z~výhod IQRF je nízka spotreba energie, ktorá umožňuje zariadeniam pracovať na batérie alebo iné zdroje energie až niekoľko rokov bez nutnosti výmeny alebo dobíjania. Ďalej sa vyznačuje svojou spoľahlivosťou a~bezpečnosťou aj v~rušných podmienkach.

    Medzi nevýhody však patrí obmedzený počet dostupných zariadení a nástrojov v~porovnaní s~ostatnými bezdrôtovými technológiami. Nedosahuje vysoké prenosové rýchlosti a je aj drahšia, kvôli potrebe špeciálnych zariadení a modulov.
\end{itemize}

Všetky tieto prenesené dáta musia byť spracované a niekde uložené, bližšie o~týchto úkonoch v~podkapitole~\ref{iot-data}.


\subsection{Uloženie a spracovanie dát}\label{iot-data}
Snímače IoT generujú údaje postupne alebo na aktiváciu vonkajšej udalosti. 
Tie sa musia zhromažďovať, agregovať, analyzovať a vizualizovať, aby sa získali užitočné informácie \cite{s20216076}.


To sa môže diať na troch IoT úrovniach — \emph{Cloud, Fog, Edge}:
\begin{itemize}
    \item \textit{\textbf{Edge computing}} spracováva dáta mimo centralizovaného úložiska, uchováva informáciu na lokálnych častiach siete — koncových zariadeniach a bránach\footnote{Anglicky využívaný názov \emph{gateway}.}. 
    To znamená najnižšiu latenciu a okamžitú odpoveď na dáta. 
    Tento prístup umožňuje vykonať výpočty a uložiť určitý (len obmedzený) objem údajov. 
    Zvyčajne má voľne prepojenú konštrukciu, v~ktorej okrajové uzly pracujú nezávisle s~údajmi.~\cite{teamDigetum_2022}
    
    Výpočty na okraji IoT sú optimálnym riešením pre malé okamžité operácie, ktoré sa musia spracovávať rýchlosťou milisekundy. 
    Keď sa súčasne uskutočňuje veľa malých operácií, ich vykonávanie na miestnej úrovni je rýchlejšie a lacnejšie.~\cite{teamDigetum_2022}
    
    \item \textit{\textbf{Fog computing}} funguje ako medzivrstva medzi tradičným centralizovaným systémom uchovávania údajov a koncovými zariadeniami. 
    \emph{Fog} rozširuje \emph{cloud} a približuje výpočty a uchovávanie údajov k~okraju. 
    \emph{Fog} pozostáva z~viacerých uzlov (\emph{Fog nodes}) a vytvára miestnu sieť, ktorá z~nej robí decentralizovaný ekosystém. ~\cite{teamDigetum_2022}
    
    Keď sa údaje dostanú do \emph{fog} vrstvy, uzol sa rozhodne, či ju spracovať lokálne alebo ju poslať do \emph{cloudu}. 
    K~údajom je preto možné pristupovať offline, pretože niektoré ich časti sa uchovávajú aj na miestnej úrovni.~\cite{teamDigetum_2022}
    
    \emph{Fog computing} je užitočný, keď je pripojenie na internet nestabilné. 
    Napríklad v~pripojených vlakoch môže \emph{fog} získať lokálne uložené údaje v~oblastiach, kde nie je možné udržiavať pripojenie na internet. 
    Umožňuje tiež implementáciu spracúvania údajov v~miestnej sieti bližšie k~okrajovým uzlom, čo je dôležité pre časovo citlivé operácie a~analýzu údajov v~reálnom čase.~\cite{teamDigetum_2022}

    
    \item \textit{\textbf{Cloud computing}} je momentálne štandardom uchovávania údajov na internete. 
    Je to forma výpočtov, pri ktorých sa údaje uchovávajú na viacerých serveroch a sú prístupné online z~akéhokoľvek zariadenia.~\cite{teamDigetum_2022}
    
    \emph{Cloud} je centralizované úložisko, ktoré sa nachádza ďalej od koncových bodov ako akékoľvek iné úložisko. 
    To vysvetľuje najvyššiu latenciu, náklady na šírku pásma a~sieťové požiadavky. 
    Na druhej strane je \emph{cloud} mocným globálnym riešením, ktoré dokáže účinne zvládnuť obrovské množstvo údajov zapojením viac výpočetných zdrojov a~získaním väčšieho serverového miesta.
    Funguje skvele pre analýzu veľkých údajov, dlhodobé uchovávanie údajov a analýzu historických údajov.~\cite{teamDigetum_2022}
\end{itemize}
