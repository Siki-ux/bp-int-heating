\chapter{Problematika kúrenia}\label{kur}
Na základe štatistík z~\cite{scitanie_2021}, väčšina obyvateľov žije v~bytoch alebo bytových domoch. 
V~nich sa hlavne využíva centrálne kúrenie~\cite{kurenie_byt}. 
Preto prieskum a riešenie tejto práce je zamerané na byty a bytové domy.
Toto tvrdenie potvrdila aj analýza z~podkapitoly~\ref{analyza-user}.

V~dnešnej dobe, kvôli rastúcej cene energií, je kritickým problémom efektívnosť vykurovania. 
Každý sa snaží čo najviac ušetriť na energiách a vykurovacích palivách. 
Volíme na~to rôzne taktiky~\cite{triky_kurenia_2020}:
\begin{itemize}
    \item Pre každú izbu volíme inú, vhodnú teplotu.
    \item Kúrenie nevypíname úplne.
    \item Utesníme a postaráme sa o~okná.
    \item Dodržujeme zásady vetrania.
    \item Nezakrývame radiátory.
    \item Manuálne regulujeme kúrenie.
    \item Udržujeme výhrevné telesá v~poriadku.
\end{itemize}
Niektoré z~týchto taktík sú ale dosť únavné a neefektívne.
Preto túto činnosť môžeme prenechať šikovným zariadeniam. 
Niekoľko takýchto zariadení na ovládanie teplôt sú ukázané v~podkapitole~\ref{solutions}. 
Ďalším krokom v~šetrení energiami je používať najefektívnejší spôsob kúrenia. 
Tieto spôsoby sú opísané v~podkapitole~\ref{heating}.


\section{Spôsoby kúrenia}\label{heating}
Jedným z~najdôležitejších aspektov domova je jeho teplo. 
Teplo musí ale niekde vznikať. 
Poznáme dva základné systémy~\cite{kubba2012handbook}. 
Jedná sa o~\emph{lokálny systém} obsiahnutý v~podkapitole \ref{local} a \emph{ústredný systém} s~topnými telesami, opísaný v~podkapitole \ref{central}.

\subsection{Lokálny systém}\label{local}
Jedná sa o~výhrevné zariadenie generujúce teplo priamo v~miestnosti, ktorú chceme vyhrievať~\cite{purkynve1891topeni}. 
Na odovzdávanie tepla sa používa konvekčný alebo sálavý spôsob~\cite{lokal}.
Medzi modernejšie verzie tohto vykurovania patrí teplovzdušný komín.

Medzi takéto zariadenia patria napríklad kachľové pece alebo krby. 
Tie môžu byť otvorené alebo zatvorené podľa prístupu k~ohňu. 
Ďalej ich vieme rozdeliť aj podľa výhrevného paliva na plynné, tuhé alebo aj elektrické. 
Medzi elektrické patria teplovzdušné ohrievače.~\cite{vstajer2020elektricke}

Takýto spôsob vyhrievania miestností splní účel a miestnosť vyhreje, ale jeho spotreba palív je veľmi vysoká a neefektívna~\cite{pichova2012elektricke}. 
Zároveň aj účinnosť tohto systému je nižšia~\cite{pichova2012elektricke}. 
V~prípade krbu a kachľovej pece je obťažná aj manuálna regulácia a vzniká potreba udržiavať oheň.


\subsection{Ústredný systém}\label{central}
Poskytuje teplo viacerým priestorom v~budove z~jedného hlavného zdroja tepla. 
Teplo sa získava meničmi energie na teplo. 
To sú napríklad kotly, bojlery a tepelné čerpadlá.~\cite{centralHeating}

Primárnymi zdrojmi energie môžu byť palivá ako uhlie, drevo, ropa, petrolej, zemný plyn alebo elektrina.

Ak sa vytápa jedna bytová jednotka alebo jedno poschodie budovy, hovorí sa o~\emph{etážovom vykurovaní}. 
Dnes sa ale častejšie používa \emph{pružné etážové vykurovanie}, kde sa do trubiek pridávajú čerpadlá a tým pádom je možné vyhriať viac poschodí.~\cite{etaz}

Teplo ale podľa zdroja~\cite{teplarne} môže vznikať aj vo vzdialenom zdroji, napríklad v~teplárňach.
Takýto systém sa nazýva \emph{diaľkové vykurovanie}. 
V~takom prípade sa prenáša teplonosná látka priamo do budov, kde sa ďalej delí na poschodia, byty, miestnosti. 
Tento spôsob je najčastejšie využívaný v~panelákoch, vežiakoch a iných veľkých budovách.

Podľa zdroja~\cite{centralHeatingPlu} je výhodou tohto systému centrálne ovládanie vykurovania, ktoré uľahčí reguláciu teplôt teplonosných látok. 
Zároveň sa v~prípade elektrických alebo plynových zdrojov tepla naskytuje možnosť automatizácie, nakoľko modernejšie kotly a boilery sú programovateľné. 
To všetko zvyšuje účinnosť a efektívnosť vykurovania. 
V~prípade diaľkového vykurovania zaniká aj potreba starať sa o~zdroj tepla, nakoľko to zabezpečuje dodávateľ.

Zdroj~\cite{centralHeatingPlu} ďalej spomína, že vzniknuté teplo z~ústredného topenia musí byť rozdelené do~miestností domácností a následne distribuované v~danej miestnosti. 
Medzi dva najznámejšie vykurovacie telesá patrí \emph{podlahové kúrenie} a \emph{radiátorové kúrenie}.

\subsection*{Podlahové kúrenie}
Toto vykurovacie teleso je skryté v~podlahe, kde sa rúrkami pod podlahou šíry teplo. 
Tieto trubky sú rozložené po celej podlahovej ploche miestnosti. Vyhrieva sa žiarením, konvekciou a vedením. 
Môže byť buď teplovodné, teplovzdušné alebo elektrické.~\cite{tuffin_2018}

Značnou výhodou tejto varianty je skutočnosť, že sa používa nižšia teplota (35 – 45°C). 
Vďaka tomu je možné ušetriť na energiách a dokonca je možné uvažovať o~použití obnoviteľných zdrojov energie, ako sú napríklad solárne zdroje. 
Pri takýchto nižších teplotách a~rozložení vykurovacieho telesa po celej ploche podlahy je prúdenie vzduchu nižšie, to má za~následok rovnomernejšie rozloženie tepla. 
Navyše absencia radiátora nám zvyšuje úžitkovú plochu miestnosti a slobodu pohybu.~\cite{olesen2002radiant}

Regulácia teploty je väčšinou termostatom, kde sa na základe zvolenej teploty čerpadlami privedie do žiadanej miestnosti teplovodná látka~\cite{olesen2002radiant}. 
Viac o~termostatoch v~podkapitole~\ref{solutions}.

Používa sa v~nízkoteplotných ústredných vykurovacích systémoch. 
Tejto skutočnosti by mal zodpovedať aj zdroj tepla. 
Preto môže byť inštalácia tohto systému na niektorých miestach nemožná. 
Zároveň sa jedná o~rozsiahlu konštrukciu s~výššími nákladmi na inštaláciu.



\subsection*{Radiátorové kúrenie}
Zo zdroja~\cite{radiator_2014} vychádza, že radiátor vo svojom základe je tepelný výmenník, ktorým sa ohrieva vzduch miestnosti. 
Takže väčšina tepla je šírená prúdením vzduchu, samozrejme ale vedie teplo aj sálaním.

Ďalej zdroj~\cite{radiator_2014} spomína, že väčšinou sa v~radiátoroch používa horúca voda (50 – 70°C), v~prípade starších stavieb para z~kotla. 
Používajú sa liatinové, oceľové alebo hliníkové žiabrové radiátory, ktoré svojím tvarom podporujú prúdenie okolitého vzduchu. 
Zaberajú ale veľa miesta a obsahujú veľké množstvo vody, čo spomaľuje rozohrievanie. 
Preto sa často začínajú používať doskové zvárané radiátory z~oceľových alebo hliníkových plechov. 
Do~popredia sa taktiež dostávajú aj vertikálne a trubkové radiátory, ktoré strácajú na účinnosti pre viac estetické atribúty.

Horúca voda vstupuje do radiátorov. 
Ten je vybavený kohútikom alebo termostatickým ventilom, hlavicou. 
Tou sa dá regulovať prívod tepla, viac o~nich v~podkapitolách~\ref{TRV}~a~\ref{smart-termo}. 
Ochladená voda opúšťa radiátor.

Tepelná účinnosť záleží na tvare, materiále a umiestnení radiátora. 
Napriek tomu má stále nižšiu účinnosť ako podlahové kúrenie. Všeobecne sa ale jedná o~najčastejšie vykurovacie teleso.


\section{Spôsoby ovládania vykurovania}\label{solutions}
Ovládať centrálne vykurovanie sa dá niekoľkými spôsobmi~\cite{hometree_2022}. Tri z~nich sú opísane v~podkapitolách \ref{TRV}, \ref{smart-termo} a \ref{smart-home}.
Jedná sa o~existujúce produkty, ktoré sú zakúpiteľné na našom trhu a ponúkajú určité riešenie problematiky regulácie vyhrievania. 
Všetky tieto spôsoby k~svojej činnosti využívajú nejaké termostaty, ktoré sú vysvetlené a rozdelené v~podkapitole~\ref{termostat}.
Sú však sprevádzané problémom hysterézie, ktorý je stručne opísaný v~podkapitole~\ref{Hyster}.

\subsection{Termostat}\label{termostat}
Termostat je zariadenie, ktoré buď ovláda, meria alebo ovláda a meria teplotu fyzického systému~\cite{thermostat_definition_meaning}. 
Vykonáva akcie ako zapnutie, vypnutie a reguláciu toku tepla aby udržalo požadovanú teplotu fyzického systému~\cite{thermostat_definition_meaning}. 
Najčastejšie využitie má vo vykurovaní budov, ústrednom kúrení, klimatizácii, \emph{HVAC systémoch\footnote{Z anglickeho \emph{Heating, ventilation, and air conditiong}.}}, ohrievačoch vody, kuchynskej rúry, chladničkách a iných.
Termostaty využívajú rôzne spôsoby merania teplôt a aktivácie ovládacích operácii:
\begin{itemize}
    \item \textbf{Mechanické termostaty} podľa zdroja~\cite{w_2021}, využívajú bimetalického pásiku, ktorý reaguje na zmeny teplôt mechanickým posunom. 
    Tým aktivujú ovládanie ohrievacieho alebo chladiaceho zdroja. 
    Tie sa využívali v~domácom ústrednom kúrení používajúcom vodu alebo paru ako teplonosnú látku. 
    Medzi ďalšie mechanické termostaty sa ďalej radí voskový motor. 
    Je to lineárne ovládacie zariadenie, ktoré premieňa tepelnú energiu na mechanickú energiu využívaním fázovej zmeny voskov~\cite{ward1976anatomy}. 
    Počas topenia vosk typicky zväčší svoj objem o~5 – 20\% ~\cite{mozes1983paraffin}. 
    Vo voskovom motore sa môžu použiť vosky extrahované z~rastlinnej hmoty ale aj vysoko rafinované uhľovodíky. 
    Funguje na základe expanzie uzavretého obsahu vosku, ktorý sa vplyvom tepla expanduje a poháňa piest smerom von objemovým posunom~\cite{cdn_2020}. 
    Najčastejšie použitie má v~automobilovom priemysle, práčkach a umývačkách, zmiešavacích ventiloch pri \emph{HVAC systémoch} a termostatických radiátorových ventiloch. 
    Termostatické radiátorové ventily sú ďalej spomenuté v~podkapitole \ref{TRV}.
    \item \textbf{Elektricko–analógové termostaty} pozostávajú z~využitia mechanických termostatov. 
    Poznáme bimetalické prepínané termostaty, ktoré za pomoci dvoch bimetalických pásikov spínajú obvody a simulujú stavy zapnuté a vypnuté~\cite{w_2021}. 
    Patria sem aj termostaty napájané termočlánkami, ktoré získavajú milivolty. 
    To je postačujúca elektrická sila na poháňanie plynového ventilu s~nízkym výkonom~\cite{formisano_2022}. 
    Ten môžeme použiť na~ovládanie prísunu paliva do horáku.
    
\item \textbf{Elektronické termostaty} sú ovládané termistormi\footnote{Rezistor meniaci odpor na základe teploty.} alebo inými polovodičovými senzormi~\cite{challi_heating_2020}. 
Tie spracujú teplotu a zmenia ju na elektrický signál, ktorým sa ovláda vykurovací alebo ochladzovací systém~\cite{challi_heating_2020,robinson_2022}. 
Na meranie sa nepoužívajú žiadne pohyblivé súčiastky. 
Elektronickým termostatom na napájanie nestačia termočlánky ale musia byť napájane aspoň už z~batérii~\cite{robinson_2022}. 
Najčastejšie sa jedná o~digitálne termostaty umiestnené na stenu s~obrazovkou prípadne dotykovým ovládaním a nastavením denných a týždenných režimov~\cite{robinson_2022}. 
Používajú sa v~systémoch \emph{HVAC} a \emph{Smart Home}, o~ňom viac v~podkapitole \ref{smart-home}.
\end{itemize}

\subsection{Hysterézia}\label{Hyster}
Podľa zdroja~\cite{cadence_2022} všetky termostaty pri ovládaní teploty sprevádza problém hysterézie. 
Tá~spôsobuje oneskorenie odozvy na regulačný zásah. 
Napríklad keď ventil uzavrie prívod teplej vody do radiátoru, teplá voda v~miestnosti ďalej rastie, pretože radiátor je plný horúcej vody. 
Naopak, pri poklese teploty sa ventil otvorí. Bude však ešte dlho trvať než sa radiátor naplní teplou vodou. Vo výsledku bude teplota silno kolísať. 
Preto sa v~prípade potreby používajú zložitejšie systémy na báze PID\footnote{Z anglického proportional–integral–derivative.} regulátoru, kde okamih regulačného zásahu je nastavený tak, aby systém udržal optimálny priebeh teplôt bez výrazného kolísania.

\subsection{Termostatické ventily}\label{TRV}
Zdroj~\cite{castrads_2022} podrobne popísal túto technológiu a je z~neho čerpaný obsah pre celú podkapitolu.
Termostatický radiátorový ventil (ďalej skratka TRV) ako samoregulačný ventil namontovaný na radiátor teplovodného vykurovacieho systému na reguláciu teploty miestností zmenou prietoku teplej vody do radiátora.

TRV s~voskovým motorom obsahuje zátku, typicky vyrobenú z~vosku. 
Zátka sa rozširuje alebo zmršťuje s~okolitou teplotou. 
Je pripojená ku kolíku a ten zas k~ventilu. 
To nám vytvára spomínaný voskový motor opísaný v~podkapitole \ref{termostat}. 
Ventil sa postupne zatvára so zvyšujúcou sa teplotou okolia, čím sa obmedzuje množstvo teplej vody vstupujúcej do~radiátora. To umožňuje nastaviť maximálnu teplotu pre každú miestnosť. V~prípade odmontovania hlavice sa ventil otvorí a radiátor bude stále horúci.
Keďže ventil funguje na základe snímania teploty vzduchu, ktorý ho obklopuje, je dôležité zabezpečiť aby nebol zakrytý materiálom, napríklad závesmi.
Dôležité je si ešte uvedomiť že sníma iba teplotu v~blízkosti radiátora, ktorá môže byť značné iná v~iných častiach miestnosti.

Medzi možný problém sa ešte radí kombinovanie TRV s~inými termostatmi. 
Pretože TRV je nastavené aby sa vyplo pred dosiahnutím nastavenej teploty z~dôvodu hysterézie spomínanej v~podkapitole \ref{Hyster}. 
Termostat ale bude ďalej usilovať o~dosiahnutie nastavenej teploty, čo môže viesť k~vysokej teplote v~iných miestnostiach alebo ku kompletne nepredpokladateľnému chovaniu.

\subsection{Inteligentné termostatické hlavice}\label{smart-termo}
Zdroj~\cite{morris_2022} podrobne popísal túto technológiu a je z~neho čerpaný obsah pre celú podkapitolu.
Jedná sa o~elektronicky ovládané TRV, teda elektronické termostaty spomínané v~podkapitole \ref{termostat}, od ktorých aj získavajú vlastnosti a schopnosti, ako napríklad programovateľnosť na dni a týždne. 
Oproti konvenčným TRV teda získavajú značnú radu funkcií a ovládania čo prináša ešte lepšie energetické úspory. 

S~nástupom novších technológií hlavice získali možnosť pripojenia do IoT. 
Tým pádom už hlavica nemusí merať teplotu len v~blízkosti radiátoru ale môže mať ďalšie tepelné senzory v miestnosti. 
Takéto hlavice často potrebujú komplexnejšiu inštaláciu oproti konvenčným TRV a to napríklad pripojením k~IoT bránam a spárovaním s aplikáciami.
Najčastejšie sa využívajú bezdrôtové protokoly Zigbee alebo Z-Wave.
Niekoľko takýchto riešení je opísaných v~podkapitole \ref{analyza-solutions}.

\subsection{Smart Home}\label{smart-home}
Podľa zdroja~\cite{1382266} ide o~robustný, automatizovaný systém pre celú domácnosť. 
Tento systém monitoruje a ovláda rôzne prvky domácnosti ako napríklad: osvetlenie, audio, okná, spotrebiče, dvere, okupáciu, zabezpečenie a alarmy. 
Samozrejme monitoruje a ovláda aj teplotu v~domácnosti. 
Mimo iné nájdeme aj netradičnejšie \emph{zariadenia}, pripojiteľné do takejto domácnosti ako napríklad posteľ, vankúš, stôl, a iné.

Zdroj~\cite{1382266} ďalej popisuje, že systém je často pripojený k~internetu buď cez bránu alebo integrovanú domácu jednotku, ktorá zvládne samostatne spracúvať dáta z~množstva senzorov a na základe získanej informácie adekvátne ovládať celú domácnosť.

Tento prístup nám poskytuje možnosť ovládať všetky zariadenia z~jedného miesta alebo aplikácie. 
To zároveň pridáva na flexibilnosti.
Bezpečnosť domácnosti, vybavenej inteligentnými zariadeniami na zabezpečenie \emph{smart} domácnosti, sa značne zvýši. 
Diaľkové ovládanie všetkých spotrebičov uľahčí prístup. 
Získame štatistiky celej domácnosti a manažment nad~každým jej prvkom.~\cite{smart_home}

Vďaka obrovskému množstvu vstupných dát zo senzorov je možné získať značné informácie o~domácnosti a tým aj ušetriť veľké množstvo energii. 
To napríklad vypnutím alebo stlmením svetiel v~miestnostiach, kde sa nikto nenachádza, reguláciou vyhrievania podľa potreby, vhodnému vetraniu a inými.

Tieto systémy patria do najvyššej cenovej kategórie, nakoľko sú robustné a musia byť kúpené od jedného výrobcu. 
Pripojenie zariadení od rôznych výrobcov pod jeden systém alebo aplikáciu je často buď veľmi náročné alebo až nemožné. 
Vytvorí to nežiadúci efekt množstva brán a aplikácii spravujúcich zariadenia. 
Posledný zásadný problém, ktorý sa objaví pri takejto domácnosti je, že napriek získanej bezpečnosti pred fyzickými útokmi a~vlámaniami, sa objaví nová slabina v~podobe \emph{kyber}~útokov.~\cite{dissmart}

Jednou z~mnohých firiem, ktoré ponúkajú \emph{Smart Home} riešenie, je \textit{HomeSystem}\footnote{Stránky predajcu na \url{https://www.homesystem.sk}.}.

