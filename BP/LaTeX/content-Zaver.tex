\chapter{Záver}\label{zaver}
Cieľom tejto bakalárskej práce bolo navrhnúť a implementovať systém regulácie ústredného kúrenia so zameraním na reguláciu jednotlivých miestností. 
Tento systém mal byť diaľkovo ovládateľný platformou Logimic Smart City a mal by automaticky regulovať vykurovanie adaptívnym spôsobom. 
Tento spôsob by mal priniesť značné úspory do domácností, ktoré by boli takýmto systémom vybavené. 
Toto riešenie by sa malo oproti existujúcim riešeniam líšiť hlavne tým, že si do systému zadáme požadovanú teplotu miestnosti a systém bude ovládať výhrevné telesá aby požadovanú teplotu miestnosti docielil a udržal.

V~rámci teoretickej časti som si musel poriadne naštudovať problematiku kúrenia a~IoT. Nakoľko obe témy sú dosť robustné a pre mňa boli úplne nové, som ich prieskumu a samoštúdiu venovaný značný čas. Poznatkami z~kapitoly problematika kúrenia, dotazníkom, konzultáciou s~firmou Virtuálny správca budov a hlavne konzultáciami s~firmou Logimic som vyvodil požiadavky na inteligentné ovládanie kúrenia. Na základe toho som navrhol modul ovládania kúrenia pre existujúcu aplikáciu iTemp. Úpravy \emph{dashboardu} neboli nutné, nakoľko táto aplikácia mala všetky potrebné ovládacie a zobrazovacie prvky. Spomínaný modul som implementoval a nasadil do ostrého riešenia aplikácie iTemp.
Túto implementáciu som následne testoval ako v~reálnom prostredí, tak aj v~simulácii, v~ktorej som aj skúšal rôzne spôsoby a modely regulácie.

Tento systém je reálne použiteľný a momentálne ovláda jednu konkrétnu miestnosť. V~nej dosahuje dobré výsledky a udržiava požadovanú teplotu. Ďalej by bolo možné do~systému implementovať veľa ďalších funkcií, ktoré aj požadovali potencionálny zákazníci v~dotazníku. Ide napríklad o~detekciu otvoreného okna, denné, týždenné a iné programy. Prípadne by bolo možné vylepšiť algoritmus o~predikciu budúcnosti pre ešte lepšie a presnejšie ovládanie kúrenia.
