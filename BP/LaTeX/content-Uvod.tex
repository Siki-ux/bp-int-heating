\chapter{Úvod}
Každý človek ku komfortnému životu potrebuje pre neho komfortnú teplotu. 
To dosahujú zapaľovaním ohňov, prikladajú do piecok alebo napríklad ženú horúcu vodu radiátorom, aby zmiernili efekty vonkajšej zimy a vytvorili si tepelnú pohodu. 
Od ohniska v~jaskyni, ľudia vo vykurovaní urobili riadny pokrok. 
Technologický vývoj v~oblasti internetu vecí ponúka možnosť aplikovať tieto nové bezdrôtové technológie aj na túto sféru života~a~bývania. 
A~to~tak, že pripojíme aj kúrenie k~internetu s~cieľom využiť jeho sily pre uľahčenie prístupu ku~regulácii a~hlavne ušetrenia nákladov na~kúrenie. 

Cieľom je navrhnúť a implementovať inteligentný systém pre reguláciu ústredného kúrenia. 
Hlavnou úlohou systému je inteligentná časť, ktorá ovláda výhrevné telesá v~miestnostiach za účelom dosiahnutia ideálnej teploty miestnosti~a~jej udržiavanie. 
Druhá časť je ponúknuť používateľom diaľkový, prehľadný~a~intuitívny spôsob ovládania tohto systému. 
Celkovým výstupom mala byť aplikácia prístupná z~platformy ACADA aplikácie iTemp firmy Logimic. 
Táto aplikácia mala priniesť jednoduchý diaľkový spôsob ovládania teploty v~miestnosti vybavenej tepelnými senzormi a ovládateľnou výhrevnou jednotkou.

Táto práca je logicky rozdelená do niekoľkých oddelených kapitol a podkapitol. 
Úvodné kapitoly sa zameriavajú na teóriu ohľadom technológii a to hlavne na sféru internetu vecí v~kapitole \ref{Internet_veci}, nakoľko táto téma spadá do tejto kategórie. 
Ďalej pokračuje rozsiahla téma problematiky kúrenia v~kapitole \ref{kur}, ktorá vysvetľuje základné pojmy ohľadom kúrenia, spôsoby ovládania a problémy, ktoré ho sprevádzajú. 
Ďalej sú v~kapitole \ref{anal} analyzované požiadavky na systém, jeho užívateľov a existujúce riešenia. 
Na základe výsledku analýzy je v~ďalšej kapitole \ref{navrh} návrh systému. 
Ten sa venuje architektúre systému, dátovému modelu, algoritmu inteligentného kúrenia a ďalším častiam.
Kapitola \ref{impl} sa venuje popisu implementácie návrhu. 
Konkrétne sa venuje použitým zariadeniam, LoRa serveru, obsluhe zariadení a aplikácii iTemp. 
Predposledná kapitola \ref{test} sa venuje testovaniu vytvoreného systému. 
To je rozdelené do dvoch časti na testovanie v~simulácii s~fiktívnymi dátami a testovanie v~reálnom prostredí.
Posledná kapitola \ref{zaver} zhodnocuje celú prácu.